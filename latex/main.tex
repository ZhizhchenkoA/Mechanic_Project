\documentclass[12pt, a4paper]{article}

\usepackage[russian]{babel}
\usepackage[T2A]{fontenc}
\usepackage[utf8]{inputenc}
\usepackage{amsmath}
\usepackage{physics}
\usepackage{amssymb}
\usepackage{indentfirst}
\usepackage[letterpaper,top=2cm,bottom=2cm,left=3cm,right=3cm,marginparwidth=1.75cm]{geometry}



\usepackage{graphicx}
\usepackage[colorlinks=true, allcolors=blue]{hyperref}
%
%\setmainfont[Ligatures={TeX,Historic}]{Times New Roman}
\title{Периодические орбиты и их устойчивость}
\date{}
\author{}
\begin{document}
\maketitle
\tableofcontents
\newpage
\section{Вступление}

\section{Линеаризация вблизи точек Лагранжа}

Имеется система уравнений, описывающая движение тела вблизи коллинеарных точек Лагранжа  $L_1, L_2$:

\begin{equation}
\label{eq:linear}
    \begin{cases}
        \dot x = \frac{\partial H}{\partial p_x} = p_x + y\\
        \dot y = \frac{\partial H}{\partial p_y} = p_y - x\\
        \dot p_x = - \frac{\partial H}{\partial x} = p_y - x + ax\\
        \dot p_y = - \frac{\partial H}{\partial y} = - p_x - y - by
    \end{cases}
\end{equation}
где $ax, ay$ линейные части от $U_x, U_y$

Из уравнений Гамильтона следует, что 

\begin{equation}
    \begin{cases}
        ax = U_x = \ddot x - 2 \dot y\\
        by = U_y = \ddot y + 2 \dot x
    \end{cases}
    \Rightarrow
    \;\;\;
    \begin{cases}
         ax - \ddot x + 2 \dot y = 0\\
         - by + \ddot y + 2 \dot x = 0
    \end{cases}
\end{equation}

\subsection{Нахождение характеристического уравнения системы с помощью матриц}

Для удобства нахождения характеристического уравнения запишем систему в матричной форму, где каждая строка обозначает одно из уравнений системы, а столбцы~--- переменные ($x$~---  первый столбец, $y$~--- второй столбец). Характеристическое уравнений будет соответствовать определителю данной матрицы, приравненному к нулю.\\ 

\begin{displaymath}
    \begin{vmatrix}
        ax - \ddot x & 2 \dot y\\
        2 \dot x & - by + \ddot y
    \end{vmatrix}
    = 0 
\end{displaymath}
Произведём замену переменных: 
\begin{equation*}
    \begin{cases}
        y = x = 1 \\
        \dot y = \dot x = \lambda \\
        \ddot y = \ddot x = \lambda^2
    \end{cases}
\end{equation*}
Перепишем матрицу с учётом этой замены и найдём её определитель

\begin{equation*}
    \begin{vmatrix}
        a - \lambda^2 & 2 \lambda\;\\
        2 \lambda & - b + \lambda^2\;
    \end{vmatrix}
\end{equation*}
Определитель равен:
\begin{align*}
   (a - \lambda^2)(- b + \lambda^2) - 4 \lambda^2 = 0\\
    - ab + b \lambda^2 + a \lambda^2 - \lambda^4 - 4 \lambda^2 = 0\\
    \boxed{
    \lambda^4 + (4 - a - b) \lambda^2 + ab = 0
    }
\end{align*}

\subsection{Характеристическое уравнение с помощью производных высших порядков}

Для начала необходимо выразить $\dot x$
\begin{gather*}
    a \dot x - \dddot x + 2 \ddot y = 0\\
    \dot x = \frac{\dddot x - 2 \ddot y}{a}
\end{gather*}

\section{Проверка на устойчивость орбиты}

Для проверки системы на устойчивость необходимо построить матрицу Гурвица, которая имеет общий вид

\begin{equation}
    \label{gurwits}
    \begin{pmatrix}
        a_1 & a_0 & 0 & 0\\
        a_3 & a_2 & a_1 & a_0\\
        0 & a_4 & a_3 & a_2\\
        0 & 0 & 0 & 1
    \end{pmatrix}
\end{equation}
Каждый угловой определитель данной матрицы должен быть больше нуля\\
Запишем матрицу в нашем случае
\begin{equation*}
    \begin{pmatrix}
        0 & ab & 0 & 0\\
        0 & b - a - 4 & 0 & ab\\
        0 & 1 & 0 & b - a - 4\\
        0 & 0 & 0 & 1
    \end{pmatrix}
\end{equation*}

Очевидно, что все угловые определители данной матрицы равны нулю, и, следовательно, система находится на грани устойчивости.

\section{Периодические решения}

Поиск периодических орбит будет осуществляться с помощью алгоритма градиентного спуска. 
Для начала остановимся на этом алгоритме.

\subsection{Градиентный спуск}

 Метод градиентного спуска, также называемый методом наискорейшего спуск, служит для нахождения минимального экстремума функции.\footnote{Важный момент: градиентный спуск не гарантирует поиск глобального экстремума функции. В случае с двумя и более переменными возможно также нахождение седловой точки. Для поиска глобального экстремума необходимо менять начальные значения переменных и шага сходимости} 
 Градиент функции определяется как:

 \begin{equation*}
     \grad F(x, y, z) = \frac{\partial F(x, y, z)}{\partial x} \vec{i} + \frac{\partial F(x, y, z)}{\partial y} \vec{j}+ \frac{\partial F(x, y, z)}{\partial z} \vec{k} + \cdots
 \end{equation*}

В нашем случае мы используем прямоугольную Декартову систему координат, где $\vec i$ и $\vec j$ - единичные вектора, направленные параллельно осям $x$  и $y$ соответственно. 
Будем рассматривать задачу в плоскости $XOY$, поэтому координата  $z$ не используется.

Метод реализуется путём последовательных итераций, где каждое следующее значение получается:
\begin{gather*}
    x_{n + 1} = x_n - \lambda_1 \,\frac{\partial F}{\partial x}\\
    y_{n + 1} = y_n - \lambda_2 \,\frac{\partial F}{\partial y}
\end{gather*}
где $\lambda_1, \lambda_2$ --- коэффициенты сходимости, определяемые экспериментальным путём в зависимости от начальных условий. 
Начальные значения $x_0, y_0$ выбираются по такому же принципу.
% \bibliographystyle{alpha}
% \bibliography{sample}


\end{document}